\documentclass[a4paper,10pt]{article}
\usepackage[utf8]{inputenc}

%opening
\title{Discurso}
\author{}

\begin{document}

\maketitle

%En este trabajo investigamos factores de aprendizaje social que alteran la adquisici\'on de habilidades esperadas por la experiencia individual.
La inteligencia del ser humano es una medida de referencia en el campo de la inteligencia artificial.
El asombroso \'exito adaptativo del Homo Sapiens se suele explicar en t\'erminos de nuestras habilidades cognitivas.
Sin embargo, la hip\'otesis del nicho cultural sugiere que es el proceso emergente de evoluci\'on cultural el mecanismo que crea los paquetes adaptativos (herramientas e ideas), m\'as allá incluso del entendimiento de quienes los usan.

Con el objetivo de estudiar el \'exito adaptativo del ser humano, nos propusimos cuantificar el impacto de estrategias de aprendizaje social sobre las habilidades individuales.
Encontramos que las estrategias de agrupamiento proporciona un impulso en la adquisición habilidades equivalente a miles de partidas de experiencia individual.
Creemos que las propiedades emergentes de los sistemas culturales deben ser recibir mayor atenci\'on por el campo de la inteligencia artificial.


\vspace{0.5cm}

Human intelligence is a reference measures in artificial intelligence field.
The amazing adaptive success of Homo Sapiens is usually explained in terms of our cognitive abilities.
However, the cultural niche hypothesis suggests that is the cultural evolution emergeing process that creates the adaptive packages (tools and beliefs) even beyond the understanding of the individuals who use them.

In order to study human adaptive success, we set to quantify the impact of social learning strategies on individual skills.
We found that grouping strategies provides a skill boost equivalent to thousands of games of individual experience.
We believe that the emerging properties of cultural systems should be heeded by the artificial intelligence field.


\end{document}
